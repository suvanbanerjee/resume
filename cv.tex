\documentclass[a4paper,10pt]{article}
\input{glyphtounicode}
\pdfgentounicode=1
\usepackage{url}
\usepackage{parskip} 	
\RequirePackage{color}
\usepackage[usenames,dvipsnames]{xcolor}
\usepackage[scale=0.9]{geometry}
\geometry{margin=0.35in}
\usepackage{tabularx}
\usepackage{enumitem}
\newcolumntype{C}{>{\centering\arraybackslash}X} 
\usepackage{supertabular}
\usepackage{tabularx}
\newlength{\fullcollw}
\setlength{\fullcollw}{0.2\textwidth}
\usepackage{titlesec}				
\usepackage{multicol}
\usepackage{multirow}
\titleformat{\section}{\Large\scshape\raggedright}{}{0em}{}[\titlerule]
\titlespacing{\section}{0pt}{4pt}{4pt}
\usepackage[unicode, draft=false]{hyperref}
\definecolor{linkcolour}{rgb}{0,0.2,0.7}
\hypersetup{colorlinks,breaklinks,urlcolor=linkcolour,linkcolor=linkcolour}
\usepackage{fontawesome5}
% chktex 8


\begin{document}


\pagestyle{empty}
\newcommand{\resumeItem}[1]{
  \item\small{
    {#1 \vspace{-2pt}}
  }
}

\newcommand{\classesList}[4]{
    \item\small{
        {#1 #2 #3 #4 \vspace{-2pt}}
  }
}

\newcommand{\resumeSubheading}[4]{
  \vspace{-2pt}\item
    \begin{tabular*}{1.0\textwidth}[t]{l@{\extracolsep{\fill}}r}
      \textbf{#1} & \textbf{\small #2} \\
      \textit{\small#3} & \textit{\small #4} \\
    \end{tabular*}\vspace{-7pt}
}

\newcommand{\resumeSubSubheading}[2]{
    \item
    \begin{tabular*}{0.97\textwidth}{l@{\extracolsep{\fill}}r}
      \textit{\small#1} & \textit{\small #2} \\
    \end{tabular*}\vspace{-7pt}
}

\newcommand{\resumeProjectHeading}[2]{
    \item
    \begin{tabular*}{1.001\textwidth}{l@{\extracolsep{\fill}}r}
      \small#1 & \textbf{\small #2}\\
    \end{tabular*}\vspace{-7pt}
}

\newcommand{\resumeSubItem}[1]{\resumeItem{#1}\vspace{-4pt}}

\renewcommand\labelitemi{$\vcenter{\hbox{\tiny$\bullet$}}$}
\renewcommand\labelitemii{$\vcenter{\hbox{\tiny$\bullet$}}$}

\newcommand{\resumeSubHeadingListStart}{\begin{itemize}[leftmargin=0.0in, label={}]}
\newcommand{\resumeSubHeadingListEnd}{\end{itemize}}
\newcommand{\resumeItemListStart}{\begin{itemize}}
\newcommand{\resumeItemListEnd}{\end{itemize}\vspace{-5pt}}



%----------Header--------------


  \begin{tabularx}{\linewidth}{@{} C @{}}

    \Huge{Suvan Banerjee} \\[7.5pt]

    \href{https://github.com/suvanbanerjee}{\raisebox{-0.05\height}\faGithub\ suvanbanerjee} \ $|$ \ 
    \href{https://linkedin.com/in/suvanbanerjee}{\raisebox{-0.05\height}\faLinkedin\ suvanbanerjee} \ $|$ \  \ 
    \href{mailto:banerjeesuvan@gmail.com}{\raisebox{-0.05\height}\faEnvelope \ banerjeesuvan@gmail.com} \ $|$ \ 
    \href{tel:+919462122507}{\raisebox{-0.05\height}\faMobile \ +91 9462122507}
    \\
    \\
  \end{tabularx}


%----------Education-----------


  \section{Education}
    \resumeSubHeadingListStart
      \resumeSubheading
        {Dayananda Sagar College Of Engineering}{SGPA: 9.05}%
        {B.E in Information Science (VTU)}{2022 -- 2026}
        \hspace{4pt}
    \resumeSubHeadingListEnd
    \resumeSubHeadingListStart
      \resumeSubheading
        {Kendriya Vidyalaya No.1 Bikaner}{87.4\%}
        {CBSE --- Higher Secondary/Senior School}{2020 -- 2022} \\
    \resumeSubHeadingListEnd
  \vspace{-15pt}


%----------Skills--------------

\vspace{6pt}
  \section{Skills}
  \begin{tabularx}{\linewidth}{@{}l X@{}}
    \textbf{Languages:} &  \normalsize{Python, C, C++, JavaScript, Bash } \\
    \textbf{Frameworks/Technologies:}  &  \normalsize{Linux, Git, Github, Flask, FastAPI, Cryptography, Docker}\\
  \end{tabularx}
  \vspace{-5pt}


%----------Experience-----------
  
    \section{Experience}
    \begin{tabularx}{\linewidth}{ @{}l r@{} }
      \vspace{-5pt}
      \textbf{Opensource}|\textit{Open Voice OS} \\
      \multicolumn{1}{@{}X@{}}{
        \begin{itemize}
          \item Open Voice OS is an open-source voice assistant built on top of Mycroft-core.
          \item Developed entire website using Next.js, Tailwind, along with a blog site, and maintained the same.
          \item Designed 2 plugins for the voice assistant: one for fetching stock prices and a dependency-free VAD plugin.
          \item Assisted with localization, contributed to the installer script, added Zorin OS support by writing Ansible roles.
        \end{itemize}} \vspace{-7pt} \\
    \end{tabularx}    


%----------Projects------------


  \section{Projects}
    
    \vspace{7pt}
    
  \begin{tabularx}{\linewidth}{ @{}l r@{} }
  \vspace{-5pt}
    \textbf{Portfolio} \textit{} & \hfill \href{https://suvanbanerjee.github.io/}{Portfolio}\\
    \multicolumn{1}{@{}X@{}}{
    \begin{itemize}
        \item Showcase of personal and professional projects.
    \end{itemize}} \vspace{-7pt} \\

    \vspace{-5pt}
    \textbf{Project Zeta}|\textit{Flask, Supabase} & \hfill \href{https://projectzeta.vercel.app/}{Deployed} \\
    \multicolumn{1}{@{}X@{}}{
    \begin{itemize}
        \item A web platform for teachers where they can create their own ctf like challenges.
        \item Utilized Flask for backend and Supabase for database.
        \item Proof of concept for a new way of teaching cybersecurity using a gamified approach.
    \vspace{-10pt} 
        \end{itemize} 
        \vspace{-7pt} 
        }  \\

\vspace{-5pt}
    \textbf{Smart Traffic Management System} | \textit{Python, Pyfirmata, ftplib} & \hfill \href{https://github.com/suvanbanerjee/Smart-Traffic-System}{GitHub} \\
      \multicolumn{1}{@{}X@{}}{
        \begin{itemize}
            \item Utilized computer vision (yolo v8) for adaptive traffic signal control and real-time traffic density calculation.
            \item Utilized Arduino Uno to control traffic lights, enhancing traffic management and signal synchronization.
            \item Implemented WLAN data transfer using FTP to send real-time traffic density information for signal and data logging and utilized parallel computing for efficient data processing and traffic management \vspace{-10pt} 
        \end{itemize}
            }  \\

\vspace{-5pt}
    \textbf{Youtube Downloader}|\textit{Python, Pytube, GUI Libraries} & \hfill \href{https://github.com/suvanbanerjee/Youtube-Downloader/releases}{GitHub} \\
    \multicolumn{1}{@{}X@{}}{
    \begin{itemize}
        \item A simple GUI based youtube video downloader For Windows and Linux(DEB)
    \item Utilized Pytube library to download videos and audio from youtube.
    \item Implemented Downloading Playlist or Video in various resolutions in a user-friendly way.
    \vspace{-10pt} 
        \end{itemize} 
        \vspace{-7pt} 
        }  \\


\vspace{-5pt}
    \textbf{WeatherKit} | \textit{Python, API, Web Scraping} & \hfill
    \href{https://pypi.org/project/weatherkit/}{PyPi}
    \\
    \multicolumn{1}{@{}X@{}}{
    \begin{itemize}
        \item A weather forecasting tool that provides weather without API key.
    \item Utilized Meteo API and web scraping to provide accurate weather information.
    \item Developers can use address or location coordinates to get weather and forecast information.
    \vspace{-10pt} 
    \end{itemize}
        }  \\ 
  \end{tabularx}


%----------Achievements---------


  \section{Achievements and Certifications}


    \begin{tabularx}{\linewidth}{ @{}l r@{} }
      \textbf{Google Cybersecurity} & \hfill \href{https://www.coursera.org/account/accomplishments/professional-cert/GZJ7LQGCFFZF}{Coursera (Mar 2024)}
      \\[3.75pt]
      \multicolumn{2}{@{}X@{}}{An eight course professional certificate program in cybersecurity. Equipped with skills in threat detection, incident response, network security, python automation, SQL etc. }
    \end{tabularx}

    \begin{tabularx}{\linewidth}{ @{}l r@{} }
      \textbf{AI Security Fundamentals} & \hfill \href{https://security.certificates.lakera.ai/credentials/c46521d6-5748-4834-8bac-ecf7913c3ccc}{LakeraAI (Mar 2024)}
      \\[3.75pt]
      \multicolumn{2}{@{}X@{}}{AI Security Course, covering topics such as GenAI threats, cybersecurity frameworks, AI application security, LLM red teaming, and AI governance. }
    \end{tabularx}

%----------Extracurricular-----


  \section{Extracurricular}

    \begin{tabularx}{\linewidth}{ @{}l r@{} }
      \textbf{PointBlank --- Member} & \hfill Dec 2022 - Present \\[3.75pt]
      \multicolumn{2}{@{}X@{}}{PointBlank is a group of over 80 programmers who work to promote coding culture in colleges by participating in more than 10 contests and organising technical seminars}
    \end{tabularx}
    
    \begin{tabularx}{\linewidth}{ @{}l r@{} }
      \textbf{Arcis --- Member} & \hfill Dec 2022 - Feb 2023 \\[3.75pt]
      \multicolumn{2}{@{}X@{}}{Team Arcis is the Official Aero Design of Dayananda Sagar College of Engineering. They participate in the SAE Aero Design Challenge and AIAA DBF, two of the biggest student design competitions, held annually in USA.}
    \end{tabularx}



    \end{document}